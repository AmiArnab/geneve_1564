\documentclass[twocolumn,paper=a4,pagesize=pdftex,12pt,headinclude=on]{scrbook}
\areaset[1in]{5in}{11in}

\usepackage{bigfoot}
\usepackage{ragged2e}

\usepackage{fontspec}
\usepackage{xunicode}
\defaultfontfeatures{Ligatures=Historic,Contextuals=Alternate,Numbers=OldStyle,RawFeature={+ss02,+cv01,+dlig}}
\setmainfont[RawFeature={+ss02,+cv01,+ss05,+dlig},ItalicFeatures={RawFeature=+cv04,CharacterVariant=5:2}]{EB Garamond}
% Do the replacements manually in titles, to not ruin the textsc
\newfontfamily\booktitlefont[RawFeature={-ss02},LetterSpace=40,WordSpace=6]{EB Garamond}
\newfontfamily\spacedfont[RawFeature={-ss02},LetterSpace=20,WordSpace=3]{EB Garamond}
\newfontfamily\lettrinefont{EB Garamond Initials}
\newfontfamily\headerfont{EB Garamond}

\usepackage{graphicx}
\usepackage{lettrine}

\renewcommand{\LettrineFontHook}{\lettrinefont}


% Verse references
% To be adjusted
\usepackage{bibleref-french}
\renewcommand*{\BRchvsep}{.}%
\renewcommand*{\BRvsep}{,}%
\setbooktitle{Job}{Iob}%
\setbooktitle{Ps}{Pseau.}%
\setbooktitle{Jn}{Iean}%
\setbooktitle{He}{Hebr.}%
\setbooktitle{Je}{Ierem.}%
\setbooktitle{Ec}{Ecclestiasti.}%
\setbooktitle{Ac}{Act.}%
\renewcommand{\BRbooktitlestyle}{\textit}


% Environments
\usepackage{setspace}
\newenvironment{bookcomment}
  {\begin{spacing}{0.8}\itshape\scriptsize\hspace{-1em}}
  {\end{spacing}}


\newenvironment{chaptercomment}
  {%
    \setlength{\leftskip}{1em}
    \setlength{\rightskip}{1em}
    \begin{spacing}{0.8}
      \itshape\tiny\hspace{-2em}}
  {%
    \end{spacing}
    \setlength{\leftskip}{0pt}
    \setlength{\rightskip}{0pt}
  }

% Footnotes
\usepackage{alphalph}
\DeclareNewFootnote{main}[alph]
\renewcommand{\thefootnotemain}{\alphalph{\value{footnotemain}}}
\DeclareNewFootnote{chapter}[Roman]
\DeclareNewFootnote{verse}[arabic]


\setlength{\marginparwidth}{4.3em}% adjust to your document's needs

\newcommand{\fakenote}[3][\space]{%
   \par\noindent#1\textsuperscript#2\justifying#3
}


% Bible verses
\newcounter{verse}
\newcommand{\bverse}{%
  \linebreak
  \addtocounter{verse}{1}
  \theverse\quad
}

\newcommand{\bversenopar}[1][\indent]{%
   \addtocounter{verse}{1}\\#1\theverse~
}

\newcommand{\bversenonum}{%
   \addtocounter{verse}{1}
   \par
}


\usepackage{titlesec}
% Use fourier ornaments
\usepackage{fourier-orns}

% Bible books
\newcommand{\bbook}[4][]{%
  \makebox[\textwidth][c]{\includegraphics[width=6in]{#4}}
  \chapter[#1]{#2,\\\large #3\\\char"2766}
}
\titleformat{\chapter}[hang]%
   {\centering\huge}%
   {}%
   {0pt}%
   {}
\titlespacing*{\chapter}
  {0pt}{0pt}{5pt}

% Bible chapters
\newcommand{\bchapter}[1][chapitre]{%
   \setcounter{verse}{0}%
   \def\chaptertitle{#1}
   \section{}{}
   \setcounter{footnotemain}{0}
}

\renewcommand{\thesection}{\roman{section}}
\titleformat{\section}[hang]%
   {\booktitlefont\centering}%
   {\textsc{\chaptertitle\ \thesection}.}%
   {5pt}%
   {}
\titlespacing*{\section}
  {0pt}{0pt}{0pt}


% Headings
\usepackage{scrpage2}
\rofoot[]{}
\ohead[\pagemark]{\pagemark}

\pagestyle{scrplain}
\renewcommand*{\chapterpagestyle}{scrheadings}


% Paragraphs

\setlength{\parindent}{-5pt}
\setlength{\parskip}{4pt}


\linespread{0.9}
\setlength{\columnsep}{6mm}

\begin{document}

\twocolumn[
\begin{@twocolumnfalse}
\bbook{Le premier livre de Moyse}{Dict Genese.}{genese_heading}

\begin{center}
\booktitlefont\textsc{argvment.}
\end{center}
\begin{center}
\parbox{4.67in}{
\begin{bookcomment}
Ce premier livre comprend l'origine \& causes de toutes choses, principalement
 la creation de l'homme, qu'il a esté du \linebreak
 commencement, sa cheute \& relevement~: comment d'un tous ont esté
 procreés, \& pour leurs enormes pechés Dieu \linebreak
 les a consumés, par le deluge, reservé huict, dont la semence a rempli toute
 la terre. Puis il descrit les vies, faicts, reli- \linebreak
 gion, \& lignees des saints Patriarches, qui ont vescu devant la Loy~:
 Les benedictions, promesses, \& alliances du Sei- \linebreak
 gneur faictes avec iceux~: Comment de le la terre de Chanaan
 sont descendus en Egypte. Aucuns ont appelé ce livre, le \linebreak
 livre des Iustes.  Toutefois ceci a obtenu entre nos predecesseurs \& nous,
 qu'il est appelé Genese, qui est un mot Grec, \linebreak
 signifiant generation \& origine~: d'autant
 qu'en icelui est descrite l'origine \& procreation de toutes choses~: \& nom- \linebreak
 mément des Peres anciens, qui ont esté tant devant qu'apres le deluge,
 \& eu esgard à {\spacedfont\emph{\textsc{iesvs christ}}} descen- \linebreak
 du d'iceux selon la chair.
\end{bookcomment}
}
\end{center}
\end{@twocolumnfalse}
]


\bchapter

\begin{chaptercomment}
 \footnotemarkchapter{}Creation du ciel \& de la terre,
 \emph{II, 20.} \& de tout ce qui y est com- \linebreak
 prins.  \emph{3.14} De la lumiere aussi,
 \emph{26} \& de l'homme,
 \emph{28} {\addfontfeature{RawFeature=+swsh}Auquel}\linebreak
 tout est assubietti.
 \emph{2.2.} \emph{28} Dieu benit toutes ses \oe{}uvres,
 \emph{31} qu'il \linebreak
 a accomplies en six iours.
\end{chaptercomment}

\vspace{\baselineskip}

% TBD: span lettrine on 10 lines
\bversenonum \lettrine[lines=10,image=true,loversize=0.05]{D}{}%
 \footnotemarkverse{}Ieu
 \footnotemarkmain{}crea
 \footnotemarkmain{}au com \linebreak
 mence - ment
 \footnotemarkmain{}le ciel \& la terre.
\bversenopar[]Or la\linebreak
 terre eſ-\linebreak toit sans forme, \& \linebreak
 vuide, \& les tenebres estoyent sur les
abysmes~: \& l'Esprit de Dieu
 \footnotemarkmain{}estoit
 espandu par dessus les eaux.
\bverse Adonc Dieu dît,
 \footnotemarkverse{}Qu'il y ait lumie\-re.
 \footnotemarkmain{}Et la lumiere fut.
\bverse Et Dieu vid \~q la lumiere estoit bon\-ne~: \& separa la lumiere des tenebres.
\bverse Et Dieu appela la lumiere iour,\& les
 tenebres nuict. Lors fut faict le
 \footnotemarkmain{}soir \& le matin 
 du premier iour.
\bverse ¶ Puis Dieu dît,
 \footnotemarkverse{}Qu'il y ait une
 \footnotemarkmain{}eſ\-tendue entre les eaux,
 \& qu'elle separe les
 \footnotemarkmain{}eaux d'avec les eaux.
\bverse Dieu donc fit l'estendue, \& divisa\linebreak

% Notes for the page
\marginpar{\vspace{-6.0in}\tiny
   \fakenote{I}{Ce premier cha- \linebreak pitre est fort diffi- \linebreak
 cile~: \& pour cette \linebreak
 cause, il estoit de- \linebreak
 fendu entre les He \linebreak
 brieux de le lire \& \linebreak
 interpreter devant \linebreak
 l'aage de trente \linebreak ans.}
   \fakenote{a}{Fit de rien, \& \linebreak sans aucune ma- tiere.}
   \fakenote{1}{\bibleverse{Job}(38:4), \bibleverse{Ps} \linebreak
   33.6, \textit\& 89.12.,\linebreak 135.5, \bibleverse{Ec} \linebreak
   13.1, \bibleverse{Ac} 14-15, \linebreak
   \textit\& 17.14}
   \fakenote{b}{Tout premierement, \& av\~at qu'il y eut aucune
 creature, \bibleverse{Jn}(1:10).}
   \fakenote{2}{\bibleverse{He}(11:3).}
   \fakenote{c}{Le ciel \& la terre,
 les eaux, les \linebreak
 abysmes, se pren- nent ici pour vne mesme chose~: aſç. \linebreak
 pour une matiere c\~ofuse \& sans for- \linebreak
 me, \~q Dieu forma \linebreak
 \& agença apres \linebreak
 par sa Parole.}
   \fakenote{d}{Ou, se mou- \linebreak
 voit. C'est, souste- \linebreak
 noit et conservoit \linebreak
 en son estre cette \linebreak
 matiere confuse. \\
 Car il est impossi- \linebreak
 ble, \~q aucune cho- \linebreak
 se apres avoir esté \linebreak
 faictes,puisse subsi- \linebreak
 ster un seul mo- \linebreak
 ment, si Dieu ne la \linebreak
 soustient \& c\~oser- \linebreak
 ve par sa vertu, \linebreak
 \bibleverse{Ps}(130:).}
   \fakenote{e}{Cette lumiere \linebreak
 n'estoit point en- \linebreak
 core au soleil, car \linebreak
 il n'avoit pas esté \linebreak
 creé, mais estoit en \linebreak
 la main de Dieu, \linebreak
 ay\~at son ordre suc- \linebreak
 cessif avec les tene- \linebreak
 bres, pour faire le \linebreak
 iour \& la nuict \& \linebreak
 ce iusques au qua- \linebreak
 trieme iour, que \linebreak
 Dieu fit le soleil \linebreak
 pour estre ministre \linebreak
 \& dispensateur de \linebreak
 cette lumiere, avec \linebreak
 la lune \& estoilles.}
   \fakenote{3}{\bibleverse{Ps}(33:6), \textit\& 136.5.\\
                \bibleverse{Je}(10:11) \textit\& 51.15.}
   \fakenote{f}{Ici est la cause}
}
 \pagebreak
\vspace{\baselineskip}

 \noindent les eaux, qui estoyent sous
 l'estendue, d'avec celles, qui estoyent sur l'esten\-due. Et fut ainsi faict.
\bverse Et Dieu appela l'estendue, Ciel.
 Lors fut faict le soir \& le matin du second iour.
\bverse ¶ Puis Dieu dît, 
 \footnotemarkverse{}~\footnotemarkmain{}Que les eaux,
 qui sont sous le ciel,
 soyent assemblees en un lieu, \& que le sec apparoisse. Et fut ainsi faict.
\bverse Et Dieu appe\char"A749 ale sec,Terre,\& l'assem \linebreak
 blee des eaux, mers.
 Et Dieu vid que celà estoit bon.
\bverse Et Dieu dît, Que la terre produise verdure, herbe produisant semence,
 \& arbre fruictier, faisant fruict selon son espece, lequel ait sa sem\~ece
 en soy-meſ\-me sur la terre. Et fut ainsi faict.
\bverse La terre d\~oc produisit verdure, her\-be produisant sem\~ece
 selon son espece, \& arbre sans fruict, lequel avoit sa \linebreak
  semence en soymesme selon son espe- \linebreak
 ce. Et Dieu vid que celà estoit bon.
\bverse Lors fut faict le soir \& le matin du troisieme iour.
%skip j counter
% TODO: do it better
\addtocounter{footnotemain}{1}
\bverse ¶ Apres Dieu dît,\footnotemarkverse{}\,\footnotemarkmain{}Qu'il y ait lumi \linebreak
 naires en l'estendue du ciel, pour
 sepa\-rer la nuict du iour~: \& soy\~et en
 \footnotemarkmain{}signes,

\begin{flushright}
a\quad en
\end{flushright}

% Notes for the page
\marginpar{\vspace{-5.7in}\tiny
 % Continue note f from left column
 pourquoy les Hebrieux c\~omencent
 le iour naturel le soir apres le soleil couchant.
   \fakenote{g}{Ce mot d'Est\~e \linebreak
 due, compr\~ed tout ce qui
 se voit par dessus nous, t\~at en la region celeste,
 qu'elementaire.}

   \fakenote{4}{\bibleverse{Ps}(33:7).}
   \fakenote{h}{Il est ici parlé de deux manieres
 d'eaux~: asçavoir, celles q sont sous
 l'estendue, comme la mer, les fleuves,
 \& autres qui sont sur la terre \& cel- \linebreak
 les, qui sont sur l'estendue,
 comme sont les nuees plei- \linebreak
 nes d'eau
 ça haut en l'air par dessus nous.
 Dieu a mis \linebreak
 entre ces deux for \linebreak
 ces d'eaux une gr\~a \linebreak
 de estendue, qu'on \linebreak
 appelle le ciel~: de \linebreak
 là nous appelons les oiseaux du ciel.}
   \fakenote{i}{Ceci apparti\~et au sec\~od iour,
 auquel Dieu separa, \& fit apparoir la terre du milieu des eaux.}
   \fakenote{k}{Il institue un nouvel ordre
 en nature, quand il faut \& ordonne le soleil distributeur
 de cette lumiere qu'il avoit creée avant lui, \& avant la lune \& les eſ- \linebreak
 toilles.}
   \fakenote{5}{\bibleverse{Ps}(136:7)}
   \fakenote{l}{C'est pour si- \linebreak
 gnifier diverses di- \linebreak
 spositions que les corps \~iferieurs se- \linebreak
 lon l'ordre de na- \linebreak 
 ture ont des corps \linebreak
 celestes, c\~ome cau \linebreak
 ses sec\~odes ordon \linebreak
 nees de Dieu à ce- \linebreak 
 là.  En quoy tou- \linebreak
 tesfois faut fuir cu- \linebreak
 riosité \& supersti- \linebreak
 tion \~q les h\~omes \linebreak
 ont c\~otrouvee sur \linebreak
 celà.}
}

% Page 2
\pagebreak

\footnotemarkmain{} en saisons,
\footnotemarkmain{}en iours,\& en ans.
\bverse Et soyent pour luminaires au firma\-ment
du ciel, à fin de donner lumiere
sur la terre. Et fut faict ainsi.
\bverse Dieu donc fit deux grans luminai\-res:le
 plus grand luminaire pour
\footnotemarkmain{} gouverner le iour,\& le moindre
pour gou \linebreak
verner la nuict:\& les estoilles.
\bverse Et Dieu les mit en l'estendue du ciel,
pour luire sur la terre,
\bverse Et pour gouverner le iour, \& la
nuict,\& pour separer la lumiere des te \linebreak
nebres.Et Dieu vid que celà estoit b\~o.
\bverse Lors fut faict le soir \& le matin du
quatrieme iour.
\bverse ¶ En apres Dieu dît, Que les~eaux
produisent abond\~ament
\footnotemarkmain{}reptile ayant ame
vivante : \& volaille vole sur la ter\-re
envers l'estendue du ciel.
\bverse Dieu donc
\footnotemarkmain{}crea de gr\~ades baleines,
\& toute creature vivante se mouvant,
que les eaux produirent selon leur es\-pece :
\& toute volaille ayant ailes chacune selon son espece : 
\& Dieu vid \linebreak
 que celà estoit bon.
\bverse Adonc \footnotemarkmain{}il les benit, disant, Fructi\-fiez,
\& multipliez, \& remplissez les
eaux des mers:\& que la volaille se mul \linebreak
tiplie en la terre.
\bverse Lors fut faict le soir \& le matin du
cinquieme iour.
\bverse ¶ Outre Dieu dît,Que la terre produise
creature vivante selon son espe\-ce,~bestail
 \& reptile, \& animau de la
terre chacun selon leur espece. Et fut
ainsi faict.
\bverse Dieu donc fit l'animau de la terre
selon son espece, \& le bestail selon son
espece, \& tout le reptile de la terre se\-lon
son espece. Et Dieu vid que celà
estoit bon.
\bverse Outreplus Dieu dît,
\footnotemarkmain{}Faisons
\footnotemarkmain{}l'h\~o\-me
\addtocounter{footnotemain}{1}% Pas de note u
à \footnotemarkverse{}\footnotemarkmain{} nostre image,
\& selon nostre
semblance, \& qu'il ait domination sur
les poissons de la mer, \& sur les oi\-seaux
du ciel, \& sur les bestes, \& sur
toute la terre, \& sur tout reptile ram\-pant
sur la terre.
\addtocounter{footnotemain}{1}% Pas de note w
\bverse Dieu donc crea l'homme à son ima \linebreak
ge:\footnotemarkverse{}\footnotemarkmain{}il les
crea,\emph{di-ie},à l'image de Dieu:
\footnotemarkverse{}il les crea masle \& femelle.
\bverse Et Dieu les benit,\& leur dît,\footnotemarkmain{}Fructi\linebreak

% Notes de gauche
% TODO

% Page 2: deuxieme colonne
fiez,\& multipliez,\& remplissez la terre,
\& l'assubiettissez : \& \footnotemarkmain{}ayez seigneurie
sur les poissons de la mer, \& sur les oi\-seaux
du ciel, \& sur tous animaux qui se mouvent sur la terre.
\bverse D'avantage Dieu dît, Voici, ie vous
ay donné toute herbe qui produit se\-mence
qui est sur toute la terre,\& tout
arbre qui a en soy fruict d'arbre pro\-
duisant semence : \footnotemarkverse{}à fin qu'ils vous
soyent pour viande.
\bverse Mesmes aussi à tous animaux de la
terre,\& à tous oiseaux du ciel,\& à tou\-te
chose mouv\~ate sur la terre,qui a en
soy ame vivante, toute herbe verde se\-ra
pour viande. Et fut ainsi faict.
\bverse \footnotemarkmain{} \footnotemarkverse{}Et Dieu vid
tout ce qu'il avoit faict, \& voilà il estoit moult bon. Lors
fut faict le soir \& le matin du sixieme
iour.

\enlargethispage{\footskip}


% Chapitre 2
\bchapter[chap.]
\lehead[\LARGE\pagemark\quad\normalfont\headerfont Creation de l'homme. Genese.]{}
\pagestyle{scrplain}

\vspace{\baselineskip}

\begin{chaptercomment}
 \emph{2} Dieu se repose le septieme iour,
 \emph{3} \& le benit.
 \emph{8} Le iardin d'Eden. \linebreak
 \emph{9} Ce qui y est.
 \emph{10} \& ce qui en sort :
 \emph{8. 15} auquel Dieu \linebreak
 met l'homme,
 \emph{19} donnant noms aux animaux.
 \emph{18. 21} Cree la \linebreak
 femme,
 \emph{24} \& institue le mariage.
\end{chaptercomment}

\vspace{\baselineskip}

\bversenonum \lettrine[lines=3,loversize=-0.2,lraise=0.2]{L}{}Es cieux donc \& la terre furent
parfaicts, \& tout l'exercite d'i\-ceux:
\bverse \footnotemarkmain{}Car Dieu avoit acc\~opli au septie\-me
iour son \oe{}uvre qu'il avoit faicte, \linebreak
\footnotemarkmain{}\& se reposa au septieme iour de tou\-te~son
\oe{}uvre qu'il avoit faicte.
\bverse Et Dieu \footnotemarkmain{}benit le septieme iour,\& le
sanctifia: pour ce qu'en icelui il avoit
cessé de toute son \oe{}uvre, qu'il avoit
creée pour estre faicte.
\bverse Telles sont les generations du ciel
\& de la terre,quand ils furent creés,au
iour que le Seigneur Dieu fit le ciel \&
la terre,
\bverse Et tout ietton du champ dev\~at qu'il
fust en la terre, \& tout herbage du
champ devant qu'il germast:car le Sei\-gneur
Dieu n'avoit point faict \footnotemarkmain{}plou\-voir
sur la terre, \& n'y avoit homme
pour labourer la terre:
\bverse Mais une vapeur m\~otoit de la terre,
qui arroisoit tout le dessus de la terre.
\bverse Or le Seigneur Dieu avoit formé
l'homme \footnotemarkmain{}de la poudre
de la terre,\footnotemarkmain{}\&
avoit inspiré en la face d'icelui spira\-tion
de vie, \& l'homme fut faict en
ame~vivante.

\begin{flushright}
Aussi
\end{flushright}

\enlargethispage{\footskip}

\end{document}

